\documentclass[12pt]{article}
\usepackage[md]{titlesec}
\usepackage[utf8]{inputenc}
\usepackage[T1]{fontenc}
\usepackage{helvet}
\usepackage{geometry}
\usepackage{array,multirow}
\geometry{
 a4paper,
 left=10mm,
 right=10mm,
 top=10mm
 }

\let\oldsection\section
\renewcommand{\section}{\stepcounter{section}\oldsection*}
\renewcommand{\thesubsection}{A\arabic{subsection}.}
\renewcommand{\thesubsubsection}{\qquad(\alph{subsubsection})}

\usepackage{sectsty}
\sectionfont{\normalfont\fontfamily{helvet}\fontsize{12}{19}\selectfont}
\subsectionfont{\normalfont\fontfamily{helvet}\fontsize{12}{19}\selectfont}
\subsubsectionfont{\normalfont\fontfamily{helvet}\fontsize{12}{19}\selectfont}

\usepackage{amsmath}
\usepackage[makeroom]{cancel}
\usepackage{minted}

\title{\textbf{Pintos Task 1 : Scheduling \\ Design Document}}
\author{\vspace{-5ex}}
\date{\vspace{-5ex}}

\begin{document}

\maketitle

\section*{\bf Group}
@imperial.ac.uk\\
@imperial.ac.uk\\
rahul.gupta19@imperial.ac.uk\\
@imperial.ac.uk\\

\section*{\bf Preliminaries}
If you have any preliminary comments on your submission, or notes for the markers, please give them here.\\~\\
Please cite any offline or online sources you consulted while preparing your submission, other than the Pintos documentation, course text, lecture notes and course staff.\\

\begin{center}
    '\textsc{\LARGE \bf Priority Scheduling}\\    
\end{center}


\section*{\bf Data Structures}
\subsection{Copy here the declaration of each new or changed '\texttt{struct}’ or '\texttt{struct}’ member, global or static variable, '\texttt{typedef}’, or enumeration. Identify the purpose of each in roughly 25 words. \textbf{(2 marks)}}
\underline{\textbf{Answer:}}\\
TODO()
%-------------- End of Question 1 ---------------------------------------------

\subsection{Draw a diagram that illustrates a nested donation in your structure and briefly explain how this works.\textbf{(4 marks)}}
\underline{\textbf{Answer:}}\\
TODO()
%-------------- End of Question 2 ---------------------------------------------

\section*{\bf Algorithms}
\subsection{How do you ensure that the highest priority waiting thread wakes up first for a (i) lock, (ii) semaphore, or (iii) condition variable? \textbf{(3 marks)}}
\underline{\textbf{Answer:}}\\
TODO()
%-------------- End of Question 3 ---------------------------------------------
\subsection{Describe the sequence of events when a call to '\texttt{lock\_acquire()}' causes a priority donation. How is nested donation handled?
 \textbf{(3 marks)}}
\underline{\textbf{Answer:}}\\
TODO()
%-------------- End of Question 4 ---------------------------------------------
\subsection{Describe the sequence of events when '\texttt{lock\_release()}' is called on a lock that a higher-priority thread is waiting for.
 \textbf{(3 marks)}}
\underline{\textbf{Answer:}}\\
TODO()
%-------------- End of Question 5 ---------------------------------------------

\section*{\bf Synchronisation}
\subsection{How do you avoid a race condition in '\texttt{thread\_set\_priority()}' when a thread needs to recompute its effective priority, but the donated priorities potentially change during the computation? Can you use a lock to avoid the race? \textbf{(2 mark)}}
\underline{\textbf{Answer:}}\\
TODO()
%-------------- End of Question 6 ---------------------------------------------

\section*{\bf Rationale}
\subsection{Why did you choose this design? In what ways is it superior to another design you considered? \textbf{(3 marks)}}
\underline{\textbf{Answer:}}\\
TODO()
%-------------- End of Question 7 ---------------------------------------------

\setcounter{subsection}{0}
\renewcommand{\thesubsection}{B\arabic{subsection}.}

\begin{center}
    '\textsc{\LARGE \bf Advanced Scheduler}\\    
\end{center}

\section*{\bf Data Structures}
\subsection{Copy here the declaration of each new or changed '\texttt{struct}' or '\texttt{struct}' member, global or static variable, '\texttt{typedef}', or enumeration. Identify the purpose of each in roughly 25 words. \textbf{(2 marks)}}
\underline{\textbf{Answer:}}\\
TODO()
%-------------- End of Question 1 ---------------------------------------------
\section*{\bf Algorithms}
\subsection{Suppose threads A, B, and C have nice values 0, 1, and 2 and each has a '\texttt{recent\_cpu}' value of 0. Fill in the table below showing the scheduling decision, the '\texttt{priority}' and the '\texttt{recent\_cpu}' values for each thread after each given number of timer ticks: \textbf{(3 marks)}}
\begin{center}
\begin{tabular}{| *{8}{c|} }
    \hline
\multirow{2}{*}{timer ticks}    & \multicolumn{3}{c|}{recent\_cpu} & \multicolumn{3}{c|}{priority} & \multirow{2}{*}{thread to run}\\
    \cline{2-7}
    &A &B &C &A &B &C &\\
    \hline
0   &     &     &     &     &     &     &       \\
4   &     &     &     &     &     &     &       \\
8   &     &     &     &     &     &     &       \\
12   &     &     &     &     &     &     &       \\
16   &     &     &     &     &     &     &       \\
20   &     &     &     &     &     &     &       \\
24   &     &     &     &     &     &     &       \\
28   &     &     &     &     &     &     &       \\
32   &     &     &     &     &     &     &       \\
36   &     &     &     &     &     &     &       \\
    
    \hline
\end{tabular}
    \end{center}
\underline{\textbf{Answer:}}\\
TODO()
%-------------- End of Question 2 ---------------------------------------------
\subsection{Did any ambiguities in the scheduler specification make values in the table uncertain? If so, what rule did you use to resolve them?
 \textbf{(2 marks)}}
\underline{\textbf{Answer:}}\\
TODO()
%-------------- End of Question 3 ---------------------------------------------
\section*{\bf Rationale}
\subsection{Briefly critique your design, pointing out advantages and disadvantages in your design choices. \textbf{(3 marks)}}
\underline{\textbf{Answer:}}\\
TODO()
%-------------- End of Question 4 ---------------------------------------------
\end{document}
